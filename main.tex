%%%%%%%%%%%%%%%%%
% This is an sample CV template created using altacv.cls
% (v1.3, 10 May 2020) written by LianTze Lim (liantze@gmail.com). Now compiles with pdfLaTeX, XeLaTeX and LuaLaTeX.
%
%% It may be distributed and/or modified under the
%% conditions of the LaTeX Project Public License, either version 1.3
%% of this license or (at your option) any later version.
%% The latest version of this license is in
%%    http://www.latex-project.org/lppl.txt
%% and version 1.3 or later is part of all distributions of LaTeX
%% version 2003/12/01 or later.
%%%%%%%%%%%%%%%%

%% If you need to pass whatever options to xcolor
\PassOptionsToPackage{dvipsnames}{xcolor}

%% If you are using \orcid or academicons
%% icons, make sure you have the academicons
%% option here, and compile with XeLaTeX
%% or LuaLaTeX.
% \documentclass[10pt,a4paper,academicons]{altacv}

%% Use the "normalphoto" option if you want a normal photo instead of cropped to a circle
% \documentclass[10pt,a4paper,normalphoto]{altacv}

\documentclass[10pt,a4paper,ragged2e,withhyper]{altacv}

%% AltaCV uses the fontawesome5 and academicons fonts
%% and packages.
%% See http://texdoc.net/pkg/fontawesome5 and http://texdoc.net/pkg/academicons for full list of symbols. You MUST compile with XeLaTeX or LuaLaTeX if you want to use academicons.

% Change the page layout if you need to
\geometry{left=1.25cm,right=1.25cm,top=1.5cm,bottom=1.5cm,columnsep=1.2cm}

% The paracol package lets you typeset columns of text in parallel
\usepackage{paracol}

% Change the font if you want to, depending on whether
% you're using pdflatex or xelatex/lualatex
\ifxetexorluatex
  % If using xelatex or lualatex:
  \setmainfont{Roboto Slab}
  \setsansfont{Lato}
  \renewcommand{\familydefault}{\sfdefault}
\else
  % If using pdflatex:
  \usepackage[rm]{roboto}
  \usepackage[defaultsans]{lato}
  % \usepackage{sourcesanspro}
  \renewcommand{\familydefault}{\sfdefault}
\fi

% Change the colours if you want to
\definecolor{SlateGrey}{HTML}{2E2E2E}
\definecolor{LightGrey}{HTML}{666666}
\definecolor{DarkPastelRed}{HTML}{450808}
\definecolor{PastelRed}{HTML}{8F0D0D}
\definecolor{GoldenEarth}{HTML}{E7D192}
\colorlet{name}{black}
\colorlet{tagline}{PastelRed}
\colorlet{heading}{DarkPastelRed}
\colorlet{headingrule}{GoldenEarth}
\colorlet{subheading}{PastelRed}
\colorlet{accent}{PastelRed}
\colorlet{emphasis}{SlateGrey}
\colorlet{body}{LightGrey}

% Change some fonts, if necessary
\renewcommand{\namefont}{\Huge\rmfamily\bfseries}
\renewcommand{\personalinfofont}{\footnotesize}
\renewcommand{\cvsectionfont}{\LARGE\rmfamily\bfseries}
\renewcommand{\cvsubsectionfont}{\large\bfseries}


% Change the bullets for itemize and rating marker
% for \cvskill if you want to
\renewcommand{\itemmarker}{{\small\textbullet}}
\renewcommand{\ratingmarker}{\faCircle}

%% sample.bib contains your publications
\addbibresource{sample.bib}

\begin{document}
\name{E KAVYA}
%% You can add multiple photos on the left or right
% \photoL{2.5cm}{Yacht_High,Suitcase_High}

\personalinfo{%
  % Not all of these are required!
  \email{kavyareddy.minnu@gmail.com}
  \phone{9553788941}
  \location{Hasthinapuram,L.B.Nagar,Hyderabad-500079}
  \github{https://github.com/kavya-e}
  %% You MUST add the academicons option to \documentclass, then compile with LuaLaTeX or XeLaTeX, if you want to use \orcid or other academicons commands.
  % \orcid{0000-0000-0000-0000}
  %% You can add your own arbtrary detail with
  %% \printinfo{symbol}{detail}[optional hyperlink prefix]
  % \printinfo{\faPaw}{Hey ho!}[https://example.com/]
  %% Or you can declare your own field with
  %% \NewInfoFiled{fieldname}{symbol}[optional hyperlink prefix] and use it:
  % \NewInfoField{gitlab}{\faGitlab}[https://gitlab.com/]
  % \gitlab{your_id}
}

\makecvheader
%% Depending on your tastes, you may want to make fonts of itemize environments slightly smaller
% \AtBeginEnvironment{itemize}{\small}

%% Set the left/right column width ratio to 6:4.
\columnratio{0.6}

% Start a 2-column paracol. Both the left and right columns will automatically
% break across pages if things get too long.
\begin{paracol}{2}
\medskip

\cvsection{Education}

\cvevent{Bachelor of Engineering,IT}{Vasavi college of Engineering,Hyderabad}{August 2019 -- May 2023}{}
\textbullet{GPA-8.4}


\divider

\cvevent{Intermediate Education}{Sri Chaintanya Junior College,Hyderabad}{June 2017 -- March 2019}{}
\textbullet{Board : TSBIE, Percentage: 95.9}

\divider

\cvevent{Class X}{St.Joseph's School,Hyderabad}{April 2017}{}
\textbullet{Board : ICSE, Percentage: 94}
%\divider

\cvsection{Projects}

\cvevent{Project 1}{Water vending machine management system}{}{}
\begin{itemize}
\item A mini project with register and login system to allow the users to take water using smart cards instead of money. The registered user can recharge his smart card,take water, check balance and unregister whenever he wants.user details and date and time of usage will be stored in files.
\end{itemize}
\begin{itemize}
\item code written in c language
\end{itemize}




%% Switch to the right column. This will now automatically move to the second
%% page if the content is too long.
\switchcolumn


\cvsection{Strengths}

\cvtag{Hard-working}
\cvtag{Responsible}\\
\cvtag{Punctual}
\cvtag{Team Player}


\cvsection{TECHNICAL SKILLS}

\cvevent{Programming}{}{}{}
\cvtag{C}
\cvtag{Python}
\cvtag{Java}

\divider



\cvevent{Others}{}{}{}

\cvtag{Data Structures}
\cvtag{Html}



%% Yeah I didn't spend too much time making all the
%% spacing consistent... sorry. Use \smallskip, \medskip,
%% \bigskip, \vpsace etc to make ajustments.

% \divider

\cvsection{CERTIFICATIONS}

% \cvref{name}{email}{mailing address}
\cvevent{Python Data Structures}{University of Michigan}{}{}
\divider
\cvevent{Python Classes and Inheritance}{University of Michigan}{}{}

% \divider

\cvsection{OTHERS}
\cvevent{Sports}{}{}{}
\cvtag{Basketball}
\cvtag{athletics}

\divider

\cvevent{Extracurricular activities}{}{}{}
\cvtag{Member of dance club}

\cvtag{Environment Minister in school Cabinet}




\end{paracol}


\end{document}
